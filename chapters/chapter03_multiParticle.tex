\chapter{Study of Particle Multiplicity of Cosmic Ray Events using
  2\,m\,$\times$\,2\,m Resistive Plate Chamber Stack at IICHEP-Madurai}

One of the main goals of the INO Project is to collaborate with Indian
Industries in order to streamline the production and the procurement
of variouns components as well as the RPC detectors. Transfer of
technologies and experiences in between industries and research teams
is the key aspect of this effort.
An experimental setup consisting of 12 layers of glass Resistive Plate
Chambers (RPCs) of size 2\,m\,$\times$\,2\,m has been built at
IICHEP-Madurai (\ang{9;56;14.5}\,N \ang{78;00;47.9}\,E, on the surface)
to study the long term performance and stability of RPCs produced on
large scale in Indian industry. This setup has been collecting data
triggered by the passage of charged particles. The data is analysised
to understand the behaviour of the RPCs as well as the electronics
used to run and collect data from the setup. The data is also utilised
to gain knowledge of the cosmic ray muons reaching the surface of
earth. The measurement of the multiplicity of charged particles due to
cosmic ray interactions are presented here. The results are compared
with different hadronic models of the CORSIKA simulation. The data
collected near magnetic equator gives us vital information regarding
the capabilities of the simulation packages. As the current
experimental setup is located within 81\,km from INO-Site, in depth
analysis of this data gives also improves the the packages which are
being used in the Monte-Carlo simulations for this project.

The upper atmoshere of the Earth gets a large dosage of exposure of
high energy primary cosmic rays originating in outer space. These
primary cosmic rays consist of mostly protons with a smaller fraction
of higher \mbox{Z-Nuclei} elements\cite{cosmic1}. The angular
distributions of the primary cosmic rays are more or less isotropic
at the top of the earth atmosphere. The energy spectrum of the primary
cosmic rays follows the power-law spectrum, $dN/dE \propto E^{-\gamma}$,
where power-law parameter, $\gamma \sim $ 2.7. The shower of secondary particles consisting mainly of
\mbox{pions $\left(\pi^{\pm}/\pi^0\right)$} and
\mbox{kaons $\left(K^{\pm}\right)$} which are produced due the
interactions of primary cosmic rays with atmospheric nuclei.
The neutral pions mainly decay via electro-magnetic interactions,
$\pi^0 \rightarrow \gamma+\gamma$ whereas the charged pions decay to
muons and neutrinos via weak-interactions,
$\pi^+ \rightarrow \mu^+ + \nu_{\mu}$ and
$\pi^- \rightarrow \mu^- + \bar{\nu}_{\mu}$. The kaons also decay to
muons and neutrino and to pions in different branching fractions.
Most of the pions and kaons decay in flight and do not reach the
earth's surface.
The $\gamma$, $e^{\pm}$ do not reach the detector directly as they
interact with the roof of the laboratory and generate electromagnetic
showers. Only a small fraction of resultant muons decay into
electrons and neutrinos, 
$\mu^+ \rightarrow e^+ + \nu_{e} + \bar{\nu}_{\mu}$ and 
$\mu^- \rightarrow e^- + \bar{\nu}_{e} + \nu_{\mu}$. Thus, muons are the
most abundant charged particle from cosmic ray showers detected in the
present setup. These atmospheric muons are produced at high altitude
(average height of 20\,km) in the atmosphere and lose almost 2\,GeV
energy via ionisation loss in the air before reaching the ground. The 
density of charged particles (mainly muons) per unit surface area at
the earth's surface depends on the composition of primary cosmic ray,
power-law parameter ($\gamma$) as well as the model of hadronic
interactions at high energy which is not accessible in the laboratory.
