
\chapter{Thesis Summary}

The main goal of the INO-Collaboration is to investigate the neutrino
oscillation in the atmospheric neutrinos with the ICAL~detector.
The ICAL is primarily designed to observe the CC-interactions of
the $\nu_{\mu}/\bar{\nu}_{\mu}$s with the ability to distinguish between
the charged leptons, mainly $\mu^{+}$ and $\mu^{-}$,
which are being produced in the interactions. The research and
development program is ongoing keeping the ICAL detector in the mind.
Several prototype detectors have been built to study the performance
and stability of the RPC detector and electronics. The detector's
parameters like efficiencies of the RPCs, position and time
resolutions of the detector, strip~multiplicities and noise
in the detectors, etc. are calculated using these detectors.

The RPCs in ICAL detector is proposed to operate for more than 20
years. For the success of the experiment, each of the RPCs used in
this experiment will be in operation without showing any significant
degradation of performances during the period of operation. Hence, a
proper leak test has to be performed on all the glass gaps at the time
of production as well as during operation.
The method of testing gaps for leakage and quantifying the leak is
developed. The leak-test setups, both wired and wireless, are
operational and are being used at various facilities and industries
working along with INO-Collaboration. The test setups have decreased
the average time required per gap significantly. The knowledge gained
in this study also gives us more opportunity to better understand the
structural integrity of the glass RPCs against various atmospheric
parameters.

As a part of the ICAL R\&D program, a 12 layer stack of
2\,m\,$\times$\,2\,m RPCs has been operational at IICHEP, Madurai
since last few years to study the various detector properties. The
cosmic ray data acquired at this stack is studied for charged-particle
multiplicity. The charged-particle multiplicity in the obtained data
is compared with the air shower simulation. The main aim of this study
is to test the capability of the cosmic ray simulation packages. It
reflects that the current physics models of interactions at the earth
atmosphere are unable to reproduce the air showers accurately. The
earlier measurements of muon multiplicity which have similar
conclusions along with the present result can be used to improve the
parameters of the hadronic model at high energies and/or cosmic ray
spectral index.

A magnetised detector (mini-ICAL) with 10 layers of RPCs also has been
operational at IICHEP, Madurai to study the performance of electronics
equipment in the presence of magnetic field and to test the event
reconstruction algorithms. One of the motivations of the magnetised
mini-ICAL detector was to estimate the muon charge ratio at Madurai
and compare the result with the theoretical predictions which is very
near to the INO site.
The cosmic ray data collected by the detector setup is used to
calculate the charge ratio $(R)$ of the number of $\mu+$ to $\mu-$
arriving at the Earth's surface. Using the iterative Bayesian
Unfolding technique the momentum spectra is made free of detector
bias, and the charge ratio of muons is observed and compared with the
BESS-TeV'02 calculation.
From the study, it is seen that the ratio between $\mu^{+}$ and
$\mu^{-}$ more or less matches in the range of 0.8-3\,GeV.
The reconstruction of momentum beyond this energy fails due to
the the low-energy cutoff in this detector setup, the insignificant
curvature of the tracks created by the particles, poor position
resolution of RPCs and limited number of tracker layers. The result
of this study can also be used to improve the hadronic interaction
models and for better neutrino flux prediction. This study will
also intend to improve the charge and momentum sensitivity in the
ICAL detector.
