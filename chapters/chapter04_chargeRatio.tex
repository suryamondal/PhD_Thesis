
\chapter{Charge Ratio of Atmospheric Muons at IICHEP-Madurai}

As a part of the ICAL R\&D program, a magnetised detector (mini-ICAL)
with 10 layers of RPCs has been built and operational at IICHEP,
Madurai. Being a scale-down model of the ICAL detector, the mini-ICAL
is beeing studied a the prototype of the magnetised ICAL. This
prototype is mainly built to study the performance of electronics
equipment in the presence of the magnetic field and to test the event
reconstruction algorithms. The cosmic ray data collected by the
detector setup is also used to calculate the charge ratio $(R)$
of the number of $\mu+$ to $\mu-$ arriving at the Earth's surface.
The testing of the reconstruction algorithms is the motivation behind
this study. By comparing the result from cosmic ray data with extreme
air shower (EAS) simulation, this study can signify the ability of
the magnet in identifying the charge of the particle.


The dimentions of this
prototype detector are 4\,m$\times$4\,m$(=X)\times$1\,m$(=Z)$. to
study the performance of electronics equipment in the presence of
magnetic field and to test the event reconstruction algorithms. One of
the motivations of the magnetised mini-ICAL detector was to estimate
the muon charge ratio at Madurai and compare the result with CORSIKA
simulation which is very near to the INO site. The muon charge ratio
$R$ is defined as the ratio of the number of $\mu+$ to $\mu-$ arriving
at the Earth's surface. These muons are created at the upper
atmosphere due to the interaction of the high energy primary cosmic
rays and the air molecules. As the primary cosmic rays are dominated
by the positively charged particles, the production of the positive
mesons are favoured. Measurement of the muon charge ratio can be used
to improve the hadronic interaction models and for better neutrino
flux prediction.

The mini-ICAL detector operational at IICHEP, Madurai is consist of 11
layers of iron of size 4\,m\,$\times$\,4\,m\,$\times$\,5.6\,cm with
the interlayer gap of 4.5\,cm. A uniform magnetic field of 1.3\,T is
obtained in the central region of the detector by winding the copper
conductors in a similar fashion to the proposed ICAL detector. The
magnetic field in the mini-ICAL is along the Y-direction in the
central region. Ten RPCs of dimensions 174\,cm\,$\times$\,183.5\,cm
are used as the active detector. These RPCs are and placed in the
central region of the detector in between iron layers. The DAQ system
is similar to the setup discussed previously.

The reconstruction of momentum and charge of muons has been performed
in two stages. At first, an event is fitted with the equation of a
circle. The charge, rough momentum and initial direction of the muon
are estimated by observing the curvature of the fitted circle. This
information is then used in the second stage as input parameters. In
the second stage, the muon is then propagated in the detector medium
including ionisation energy loss in iron and other material.

The result of this study will also be used to improve the hadronic
interaction models and for better prediction of neutrino flux.
