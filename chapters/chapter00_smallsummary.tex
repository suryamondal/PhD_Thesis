\begin{center}
  %% \vspace*{0.5em}
  \large
  \underline{SUMMARY}
\end{center}

\normalsize

\doublespacing

The 50 kton INO-ICAL \cite{inowhite} is a proposed underground high
energy physics experiment at Theni, India (9$^\circ$57$'$\,N,
77$^\circ$16$'$\,E) to study the neutrino oscillation parameters using
atmospheric neutrinos. The Resistive Plate Chamber (RPC) has been
chosen as the active detector element for the ICAL detector,
interspersed with 5.6\,cm thick iron plates. Approximately 30000 RPC
gaps of dimension 2\,m\,$\times$\,2\,m will be placed in between the
151 layers of iron plates. The iron plates will be magnetised up-to
about 1.5\,Tesla. The detector will be housed inside a cavern under a
rock overburden of 1\,km to reduce the atmospheric muon background.
The ICAL will search mainly for $\nu_{\mu}$ induced charged current
interactions in the iron target. The primary aim of the experiment is
to determine the sign of the mass-squared difference
\mbox{$\Delta m^2_{32}$ $\left(=m^2_3-m^2_2\right)$} using matter
effects. The ICAL detector can also be used to probe the value of
\mbox{leptonic CP-phase $\left(\delta_{cp}\right)$} and last but not
the least to search for physics beyond the standard model using
neutrino oscillations. 

The INO~Project is proposed to operate at-least for 20 years.
Various tests are performed during and after production of the RPCs to
arrest the ageing of RPCs.
The RPCs are going to be operated in avalanche mode with
a gas mixture of R134a\,(95.2\%), iso-C$_4$H$_{10}$\,(4.5\%) and
SF$_6$\,(0.3\%).
During the active operation of the ICAL detector, $\sim$200,000\,litres
of the gas mixtures will be circulating inside
the 30,000 RPCs.
Any contamination leaking inside the RPCs as well as leaking of the
gas mixture outside, can affect the performance of the detector.
Due to this, a
proper leak test has to be performed on all the glass gaps at the time
of production as well as during operation and a proper gas monitoring
system for the Closed-loop System has to be implemented to detect
impurities in the gas-mixture during operation.
The leak-test setups, both wired and wireless, are operational and are
being used at various facilities and industries working along with
INO-Collaboration and with the help of the prepared document even a
novice can test a large number of RPCs in a short time. The knowledge
gained in this study also gives us more opportunity to better
understand the structural integrity of the glass RPCs against various
atmospheric parameters.

As a part of the ICAL R\&D program, a 12 layer stack of
2\,m\,$\times$\,2\,m Resistive Plate Chambers (RPCs) with an
inter-layer gap of 16\,cm has been operational at IICHEP, Madurai
since the last few years to study the cosmic ray muons. The data
obtained by this setup is also used to study the flux and angular
distribution of muons with the help of an extreme air shower (EAS)
simulation program and detector simulation program. To further
test the capability of the Simulation Packages, the charged-particle
multiplicity in the obtained data is compared with it with the air
shower simulation.
The results of the current study reflect that the current physics
models of interactions at the Earth's atmosphere are unable to
reproduce the air showers accurately. The earlier measurements of muon
multiplicity which also showed more muon
multiplicity in data along with the present result can be used to
improve the parameters of the hadronic model at high energies and/or
cosmic ray spectral index.

The study of atmospheric muon charge ratio
$\left(R_{\mu}=N_{\mu^{+}}/N_{\mu^{-}}\right)$
is important to the measurement of the neutrino flux precisely,
alongside the relevant information in the composition of the primary
cosmic rays and the different mechanisms of particle physics.
One of the main aspects of ICAL detector is to distinguish between
the $\mu^{+}$ and $\mu^{-}$ passing through the magnetised iron
medium, which in turns helps in determining the mass-hierarchy of
the neutrinos.
As a part of the ICAL R\&D program, a magnetised detector (mini-ICAL)
with 10 layers of RPCs has been built and operational at IICHEP,
Madurai situated near the INO site. Being a scale-down model of the
ICAL detector, the mini-ICAL is being studied as the prototype of
the magnetised ICAL.
The cosmic ray data collected by the
detector setup is also used to calculate the charge ratio $(R)$
of the number of $\mu^{+}$ to $\mu^{-}$ arriving at the Earth's surface.
The testing of the reconstruction algorithms is also another
motivation behind this study.
By comparing the result from cosmic ray data with extreme
air shower (EAS) simulation, this study also signifies the ability of
the magnet in identifying the charge of the particle.
From the study, it is seen that the ratio more or less matches
in the range of 0.8-3\,GeV with
BESS-TeV'02 calculation \cite{bess2002}.
A new detector setup, named as Engineering Module is going to be
built in the near future with 20 layers of RPCs where the momentum
should be reconstructed up-to $\sim$12\,GeV.
